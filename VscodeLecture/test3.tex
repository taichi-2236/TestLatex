\documentclass[dvipdfmx]{jsarticle}
% \usepackage{cprotect}

\title{練習用レポート}
\author{学籍番号 氏名}
\date{\today}

\begin{document}
\maketitle

\section{はじめに}
このレポートは練習用です.ここでは\LaTeX を例にして扱いますが,他の言語のプログラムでも十分に応用が利きます.このレポートを作成する過程でVSCodeの機能を利用し,実践を踏まえて使いこなせるようになることを期待します.

\section{行操作}
以下の文章を行操作機能を利用して編集しましょう.
\subsection{太陽系の惑星を太陽に近い側から順番に並べる}
\begin{enumerate}
  \item 海王星
  \item 金星
  \item 水星
  \item 土星
  \item 地球
  \item 火星
  \item 天王星
  \item 木星
\end{enumerate}

\subsection{「要素」を箇条書きで8個並べる}
\begin{itemize}
  \item 要素
\end{itemize}

\subsection{十二支の生き物を列挙し,それ以外をコメントアウトする}
\begin{itemize}
  % \item ネズミ
  \item カバ
  % \item ウシ
  \item ブタ
  % \item ライオン
  \item トラ
  % \item ウサギ
  \item キリン
\end{itemize}

\subsection{60点未満を削除する}
\begin{itemize}
  \item 90点
  \item 74点
  \item 82点
  \item 54点
  \item 70点
  \item 62点
  \item 43点
  \item 56点
  \item 80点
  \item 47点
  \item 65点
\end{itemize}

\subsection{以下に「はじめに」の本文をコピーして貼り付ける}

\section{エディター操作}
\subsection{新規ファイルを作成しpractice2.texと名前を付けて保存する}
\subsection{以下の設定で文書を作成して保存する}
\begin{itemize}
  \item \verb|documentclass|の\verb|style|は\verb|jsarticle|,\verb|option|は\verb|dvipdfmx|にする
  \item タイトル名を「練習用レポート2」,筆者を自分の名前,日付を今日にしてタイトルを作成する
\end{itemize}
\subsection{最後に開いていたファイルを再度開く}
\subsection{ワークスペースを移動してRoboDragonsSystemを開く}
\subsection{soccer.hと soccer.cc を開く}
\subsection{soccer.h の中の TFreekicker が含まれる行をコピーして以下に貼り付ける}
\subsection{soccer.cc の中の freekicker が含まれる行をコピーして以下に貼り付ける}
\subsection{開いている保存済みのファイルを全て閉じる}

\end{document}