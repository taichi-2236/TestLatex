%% vscode_practice2.tex
\documentclass[dvipdfmx]{jsarticle}

\title{VSCode講座 #2}
\author{名前:}
\date{作成日:\today}

\begin{document}
\maketitle

\section*{はじめに}
このレポートはVSCode講座第2回の練習用レポートです.まずは,タイトルの著者欄を自分の名前に変えましょう.今回は,〈行操作〉の練習を行います.該当するショートカットキーを駆使して以下のタスクに取り組んでください.

\section{行操作のショートカット}
\begin{itemize}
  \item 行複製
  \item 行移動
  \item 行削除
  \item 行選択
  \item 行コメントアウト
\end{itemize}

\section{行操作の演習}
以下の文章を行操作の機能を利用して編集しましょう.
\subsection{太陽系の惑星を太陽に近い側から順番に並べる}
\begin{enumerate}
  \item 海王星
  \item 金星
  \item 水星
  \item 土星
  \item 地球
  \item 火星
  \item 天王星
  \item 木星
\end{enumerate}

\subsection{「要素」を箇条書きで8個並べる}
\begin{itemize}
  \item 要素
\end{itemize}

\subsection{十二支の生き物を列挙し,それ以外をコメントアウトする}
\begin{itemize}
  \item カバ
  % \item ネズミ
  \item ブタ
  % \item ウシ
  \item トラ
  % \item ライオン
  \item キリン
  % \item ウサギ
\end{itemize}

\subsection{60点未満を削除する}
\begin{itemize}
  \item 90点
  \item 74点
  \item 82点
  \item 54点
  \item 70点
  \item 62点
  \item 43点
  \item 56点
  \item 80点
  \item 47点
  \item 65点
\end{itemize}

\subsection{以下に「はじめに」の本文をコピーして貼り付ける}


\end{document}